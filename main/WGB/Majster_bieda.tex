%%
%% Author: bartek.rydz
%% 24.05.2018
%%

% Preamble
\tytul{Majster Bieda}{sł. Wojciech Bellon}{Wolna Grupa Bukowina}
\begin{text}
    Skąd przychodził, kto go znał\\
    Kto mu rękę podał kiedy\\
    Nad rowem siadał, wyjmował chleb\\
    Serem przekładał i dzielił się z psem\\
    Tyle wszystkiego, co sobą miał\\
    Majster Bieda

    Czapkę z głowy ściągał, gdy\\
    Wiatr gałęzie chylił drzewom\\
    Śmiał się do ognia i śpiewał do gwiazd\\
    Drogą bez końca co przed nim szła\\
    Znał jak pięć palców, jak szeląg zły\\
    Majster Bieda

    Nikt nie pytał skąd się wziął\\
    Gdy do ognia się przysiadał\\
    Wtulał się w krąg ciepła jak w kożuch\\
    Zmęczony drogą wędrowiec boży\\
    Zasypiał długo gapiąc się w noc\\
    Majster Bieda

    Aż nastąpił taki rok\\
    Smutny rok, tak widać trzeba\\
    Nie przyszedł Bieda zieloną wiosną\\
    Miejsce, gdzie siadał, zielskiem zarosło\\
    I choć niejeden wytężał wzrok\\
    Choć lato pustym gościńcem przeszło\\
    Rudymi liśćmi jesienną schedą\\
    Wiatrem niesiony popłynął w przeszłość /x3\\
    Majster Bieda
\end{text}
\begin{chord}

\end{chord}