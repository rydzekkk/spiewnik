\tytul{Lekcja historii klasycznej}{sł. muz. Jacek Kaczmarski}{Jacek Kaczmarski}
\begin{text}
    \vin Gallia est omnis divisa in partes tres\\
    \vin Quarum unam incolunt Belgae aliam Aquitani\\
    \vin Tertiam qui ipsorum lingua Celtae nostra\\
    \vin Galli appellantur\\
    \vin Ave Caesar morituri te salutant!

    Nad Europą twardy krok legionów grzmi\\
    Nieunikniony wróży koniec republiki\\
    Gniją wzgórza galijskie w pomieszanej krwi\\
    A Juliusz Cezar pisze swoje pamiętniki

    \vin Gallia est omnis...

    Pozwól, Cezarze, gdy zdobędziemy cały świat\\
    Gwałcić, rabować, sycić wszelkie pożądania\\
    Proste prośby żołnierzy te same są od lat\\
    A Juliusz Cezar milcząc zabaw nie zabrania

    \vin Gallia est omnis...

    Cywilizuje podbite narody nowy ład\\
    Rosną krzyże przy drogach od Renu do Nilu\\
    Skargą, krzykiem i płaczem rozbrzmiewa cały świat\\
    A Juliusz Cezar ćwiczy lapidarność stylu!

    \vin Gallia est omnis...
\end{text}
\begin{chord}
    C G\\
    d E\\
    a\\
    F\\
    F C G C

    C G\\
    d E\\
    a F\\
    F C G C\\
    \hfill\break

    C G\\
    d E\\
    a F\\
    F C G C\\
    \hfill\break

    C G\\
    d E\\
    a F\\
    F C G C
\end{chord}
