 %%
%% Author: EL PROFESOR
%% %27.09.2020
%%
\tytul{Warchoł}{Jacek Kaczmarski}{Jacek Kaczmarski}
\begin{text}
    \hfill\break
    Warchoł! - krzyczą - nie zaprzeczam\\
    Tylko własnym prawom ufam\\
    Wolna wola jest człowiecza\\
    Pergaminów nie posłucha\\
    Boska ręka w tym, czy diabla\\
    Szpetnie to, czy właśnie pięknie\\
    Wola moja jest jak szabla\\
    Nagniesz ją za mocno - pęknie\\
    A niewprawną puścisz dłonią\\
    W pysk odbije stali siła\\
    Tak się naucz robić bronią\\
    By naturą swą służyła

    \vin Sprawa ze mną - jak kraj ten stara\\
    \vin I jak zwykle on - byle jaka\\
    \vin Nie zrobili ze mnie janczara\\
    \vin Nie uczynią też i dworaka\\
    \vin Wychwalali zasługi i cnoty\\
    \vin Podsycali pochlebstwem wady\\
    \vin A ja służyć nie mam ochoty\\
    \vin Warchołowi nikt nie da rady!

    Lubię tany, pełne dzbany\\
    Sute stoły i tapczany\\
    Płeć nadobną - niesurową\\
    I od święta Boże Słowo\\
    Lecz ni ksiądz, ni okowita\\
    Piekłem straszy, niebem nęci\\
    Ani żadna mnie kobita\\
    Wokół palca nie okręci\\
    Mój ból głowy, moja skrucha\\
    Moje kiszki, moja franca\\
    Moja wreszcie groza ducha\\
    Gdy Kostucha rwie do tańca!

    \vin Sprawa ze mną - jak kraj ten stara\\
    \vin I jak zwykle on - byle jaka\\
    \vin Nie zrobili ze mnie janczara\\
    \vin Nie uczynią też i dworaka\\
    \vin Wychwalali zasługi i cnoty\\
    \vin Podsycali pochlebstwem wady\\
    \vin A ja służyć nie mam ochoty\\
    \vin Warchołowi nikt nie da rady!

    Jakbym ja był człowiek z wosku\\
    W rękach wodzów, niewiast, klechów\\
    Mógłbym ich zostawić troskom\\
    Cały ciężar moich grzechów\\
    Ale znam tych stróżów mienia\\
    Sędziów sumień, prawdy zakon\\
    Spekulantów odkupienia\\
    Bo znam siebie - jako tako\\
    Ulepiony i pokłuty\\
    Niepotrzebny będę więcej\\
    Rzucą w ogień dla pokuty\\
    I umyją po mnie ręce

    \vin Sprawa ze mną - jak kraj ten stara\\
    \vin I jak zwykle on - byle jaka\\
    \vin Nie zrobili ze mnie janczara\\
    \vin Nie uczynią też i dworaka\\
    \vin Zły? - być może. Dobry? - a czemu?\\
    \vin Nie tak wiele znów pychy we mnie\\
    \vin Dajcie żyć po swojemu - grzesznemu\\
    \vin A i świętym żyć będzie przyjemniej!\\
    \vin Zły? - być może. Dobry? - a czemu?\\
    \vin Nie tak wiele znów pychy we mnie\\
    \vin Dajcie żyć po swojemu - grzesznemu\\
    \vin A i świętym żyć będzie przyjemniej!
\end{text}
\begin{chord}
    \textit{Capo III}\\
    A\\
    A E4 E\\
    A\\
    A E A\\
    a\\
    a E4 E\\
    a\\
    A E a\\
    F a\\
    F a\\
    B\\
    E

    C F C G\\
    C F C G\\
    C F C d\\
    a G a d a G\\
    C F C G\\
    C F C G\\
    C F C d\\
    a G a d a G

    A\\
    A E4 E\\
    A\\
    A E A\\
    a\\
    a E4 E\\
    a\\
    a E a\\
    F a\\
    F a\\
    B\\
    E

   C F C G\\
    C F C G\\
    C F C d\\
    a G a d a G\\
    C F C G\\
    C F C G\\
    C F C d\\
    a G a d a G

    A\\
    A E4 E\\
    A\\
    A E A\\
    a\\
    a E4 E\\
    a\\
    a E a\\
    F a\\
    F a\\
    B\\
    E


    C F C G\\
    C F C G\\
    C F C d\\
    a G a d a G\\
    C F C G\\
    C F C G\\
    C F C d\\
    a G a d a G\\
    C F C G\\
    C F C G\\
    C F C d\\
    a E a 
\end{chord}