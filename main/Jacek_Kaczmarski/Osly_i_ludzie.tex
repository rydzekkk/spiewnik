%%
%% Author: bartek.rydz
%% 19.02.2019
%%
% Preamble
\tytul{Osły i ludzie}{sł.: Jacek Kaczmarski wg Franciszka Goi, muz.: Przemysław Gintrowski 1980}{Jacek Kaczmarski}
\begin{text}
    Dosiadł mnie osioł okrakiem\\
    I rykiem obwieścił donośnym\\
    Że na wierzchu takim\\
    Będzie jeźdźcem wolności\\
    Wbił mi w żebra ostrogi\\
    Wepchnął do ust kostkę cukru\\
    Chwostem tnąc w poprzek nogi\\
    Zmusił bez trudu to truchtu

    Zaduch ośli mnie dusi\\
    Jestem kloaką jeźdźca\\
    Bo co spod ogona wypuści\\
    Na moich gromadzi się plecach\\
    Małpy śpiewają mu pieśni\\
    I łaską zwą co jest musem\\
    On ryczy nad uchem: nie śpij!\\
    I już nie truchtem lecz kłusem

    Mija nas inny wierzchowiec\\
    Z wysiłku ogłuchł i oślepł\\
    Osioł go kopie po głowie\\
    Cień jego uszy ma ośle\\
    Nikt już nie mówi po ludzku\\
    Rykiem i śmiechem się chwalą\\
    Głaszczą i chłoszczą do skutku\\
    Kłus nasz przechodzi w galop

    Pędzę na zgiętych kolanach\\
    Więcej niż znieść mogę znoszę\\
    Aż świta myśl niesłychana\\
    I ludzką głowę podnoszę\\
    To nic że ostrogi bodą\\
    Szpicruta nad uchem gra\\
    To jeździec stworzony pod siodło\\
    I jeźdźca dosiąść się da!

    I jeźdźca dosiąść się da!
\end{text}
\begin{chord}
    d2\\
    a\\
    d2\\
    a\\
    d2\\
    a d2\\
    d2\\
    a d2

    d2\\
    a\\
    d2\\
    a\\
    d2\\
    a d2\\
    d2\\
    a d2

    d2\\
    a\\
    d2\\
    a\\
    d2\\
    a d2\\
    d2\\
    a d2

    d2\\
    a\\
    d2\\
    a\\
    d2\\
    a d2\\
    d2\\
    a d2 (a)

    a d2
\end{chord}