%%
%% Author: EL PROFESOR
%% %27.09.2020
%%
\tytul{Kantyczka z lotu ptaka}{Przemysław Gintrowski}{Jacek Kaczmarski}
\begin{text}
\footnotesize{
    \hfill\break
    Patrz mój dobrotliwy Boże na swój ulubiony ludek,\\
    Jak wychodzi rano w zboże zginać harde karki z trudem.\\
    Patrz, jak schyla się nad pracą, jak pokornie klęski znosi\\
    I nie pyta - Po co? Za co? Czasem o coś Cię poprosi:

    Ujmij trochę laski nieba! Daj spokoju w zamian, chleba!\\
    Innym udziel swej miłości! Nam - sprawiedliwości!

    Smuć się, Chryste Panie w chmurze widząc, jak się naród bawi,\\
    Znowu chciałby być przedmurzem\\
	i w pogańskiej krwi się pławić.\\
    Dymią kuźnie i warsztaty, lecz nie pracą a - skargami,\\
    Że nie taka, jak przed laty łaska Twoja nad hufcami:

    Siły grożą Ci nieczyste daj nam wsławić się, o Chryste!\\
    Kalwin, Litwin nam ubliża! Dźwigniem ciężar Krzyża!
 
    Załam ręce Matko Boska, upadają obyczaje,\\
    Nie pomogła modłom chłosta - młodzież w szranki ciała staje.\\
    W nędzy gzi się krew gorąca bez sumienia, bez oddechu,\\
    Po czym z własnych trzewi strząsa niedojrzały owoc grzechu.

    Co zbawienie nam, czy piekło! Byle życie nie uciekło!\\
    Jeszcze będzie czas umierać! Żyjmy tu i teraz!

    Grzmijcie gniewem Wszyscy Święci, handel lud zalewa boży\\
    Obce kupce i klienci w złote wabią go obroże.\\
    Liczy chciwy Żyd i Niemiec dziś po ile polska czystość;\\
    Kupi dusze, kupi ziemie i zostawi pośmiewisko...

    Co nam hańba, gdy talary\\
    Mają lepszy kurs od wiary!\\
    Wymienimy na walutę\\
    Honor i pokutę!

    Jeden naród, tyle kwestii! Wszystkich naraz - nie wysłuchasz! -\\
    Zadumali się Niebiescy w imię Ojca, Syna, Ducha...

    Co nam hańba, gdy talary...
}
\end{text}
\begin{chord}
\footnotesize{
	\textit{Capo I}\\
    a C\\
    G E^7\\
    a C\\
    G E^7

    a E^7 a C G\\
    E^7 a F a/e E^7

    a C\\
    G\\
	E^7\\
    a C\\
    G E^7

    a E^7 a C G\\
    E^7 a F a/e E^7

    a C\\
    G E^7\\
    a C\\
    G E^7

    a E^7 a C G\\
    E^7 a F a/e E^7

    fis A\\
    E Cis^7\\
    fis A\\
    E Cis^7

    fis Cis^7 fis\\
    A E\\
    Cis^7 fis\\
    D fis Cis^7
}
\end{chord}