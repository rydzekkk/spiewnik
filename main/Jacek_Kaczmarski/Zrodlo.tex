%%
%% Author: bartek.rydz
%% 17.02.2019
%%
% Preamble
\tytul{Źródło}{sł. i muz. J. Kaczmarski 1978}{Jacek Kaczmarski}
\begin{text}
    Płynie rzeka wąwozem jak dnem koleiny, która sama siebie żłobiła,\\
    Rosną ściany wąwozu, z obu stron coraz wyżej, tam na górze są ponoć równiny;\\
    I im więcej tej wody, tym się głębiej potoczy\\
    Sama biorąc na siebie cień zboczy...

    Piach spod nurtu ucieka, nurt po piachu się wije, własna w czeluść ciągnie go siła.\\
    Ale jest ciągle rzeka na dnie tej rozpadliny, jest i będzie, będzie jak była,\\
    Bo źródło\\
    Bo źródło\\
    Wciąż bije.

    A na ścianach wysokich pasy barw i wyżłobień, tej rzeki historia, tych brzegów –\\
    Cienie drzew powalonych, ślady głazów rozmytych, muł zgarnięty pod siebie wbrew sobie\\
    A hen, w dole blask nikły ciągle ziemię rozcina,\\
    Ziemia nad nim się zrastać zaczyna...

    Z obu stron żwir i glina, by zatrzymać go w biegu, woda syczy i wchłania, lecz żyje\\
    I zakręca, omija, wsiąka, wspina się, pieni, ale płynie, wciąż płynie wbrew brzegom –\\
    Bo źródło\\
    Bo źródło\\
    Wciąż bije.

    I są miejsca, gdzie w szlamie woda niemal zastygła pod kożuchem brudnej zieleni;\\
    Tam ślad, prędzej niż ten, co zostawił go, znika – niewidoczne bagienne są sidła.\\
    Ale źródło wciąż bije, tłoczy puls między stoki\\
    Więc jest nurt, choć ukryty dla oka!

    Nieba prawie nie widać, czeluść chłodna i ciemna,\\
    Niech się sypią lawiny kamieni!\\
    I niech łączą się zbocza bezlitosnych wąwozów,\\
    Bo cóż drąży kształt przyszłych przestrzeni\\
    Jak nie rzeka podziemna?\\
    Groty w skałach wypłucze,\\
    Żyły złote odkryje –\\
    Bo źródło\\
    Bo źródło\\
    Wciąż bije.
\end{text}
\begin{chord}
    e\\
    C D G\\
    a e\\
    a G^{0} H

    e\\
    C D G\\
    e\\
    a\\
    C G^{0} H

    e\\
    C D G\\
    a e\\
    a G^{0} H

    e\\
    C D G\\
    e\\
    a\\
    C G^{0} H

    e\\
    C D G\\
    a e\\
    a G^{0} H

    e\\
    e\\
    a e\\
    a e\\
    C G^{0} H\\
    a\\
    e\\
    e\\
    a\\
    C G^{0} H
\end{chord}