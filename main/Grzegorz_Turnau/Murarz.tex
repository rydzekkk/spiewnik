%%
%% Author: bartosz.rydz
%% %28.05.2018
%%
\tytul{Murarz}{}{Grzegorz Turnau}
\begin{text}
    Murarz stawiał mur.\\
    Długo stawiał mur.\\
    Mijał go co rok\\
    Dzikich ptaków sznur.

    Najpierw mierzył grunt,\\
    Grząski mierzył grunt.\\
    Stracił butów sto,\\
    Zanim zrobił to.

    Potem kopał dół,\\
    Wielki kopał dół.\\
    Ktoś powiedział, że\\
    Starczyłoby pół (starczyłoby pół!)

    Dół był długi niesłychanie\\
    I szeroki niespodzianie,\\
    A głęboki taki był,\\
    Że po nocach mu się śnił.

    Najpierw dzień po dniu,\\
    Potem długo w noc\\
    Cegieł stawiał moc\\
    I na klocu kloc.

    Jeszcze szybciej chciał,\\
    Żeby musiał się dział,\\
    Więc budował on\\
    Wraz ze wszystkich stron\\
    (ze wszystkich stron!)

    Najpierw stawiał mur od dołu,\\
    Jak się stawia zwykle mury,\\
    Potem z boku w bok pospołu,\\
    A na końcu w dół od góry.

    Zaczął pracę tu,\\
    Kiedy młodym był,\\
    Ale zeszło mu,\\
    Chociaż długo żył.

    Dzikich ptaków sznur\\
    Przelatywał tam,\\
    Gdy samotny żył\\
    I umierał sam.

    Choć nie skończył muru murarz\\
    Zanim zgasła w nim natura,\\
    I choć ciało starł na wiór –\\
    Nie utracił wiary w mur.
\end{text}
\begin{chord}

\end{chord}