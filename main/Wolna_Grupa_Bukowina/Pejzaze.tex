%%
%% Author: bartek.rydz
%% 24.05.2018
%%

% Preamble
\tytul{Pejzaże Harasymowiczowskie}{sł. i muz. Wojciech Belon, ostatnia zwrotka Tomasz Borkowski}{Wolna Grupa Bukowina}
\begin{text}
    Kiedy stałem w przedświcie a Synaj\\
    Prawdę głosił przez trąby wiatry\\
    Zasmreczyły się chmury igliwiem\\
    Bure świerki o góry wsparte\\
    I na niebie byłem ja jeden\\
    Plotąc pieśni w warkocze bukowe\\
    I schodziłem na ziemię za kwestą\\
    Przez skrzydlącą sie bramę Lackowej

    \vin I był beskid i były słowa\\
    \vin Zanurzone po pępki w cerkwi baniach\\
    \vin Rozłożyście złotych\\
    \vin Smagających się z wiatrem do krwi

    Moje myśli biegały z końmi\\
    Po niebieskich, mokrych połoninach\\
    I modliłem się złożywszy dłonie\\
    Do gór, do Madonny Brunatnolicej\\
    A gdy serce kroplami tęsknoty\\
    Jęło spadać na góry sine\\
    Czarodziejskim kwiatem paproci\\
    Rozgwieździła się Bukowina
    
    Tak umykałem pod żaglem słońca\\
    W uśpionych dolin puste ogrody\\
    Pomiędzy chyże w podniebnych trawach\\
    Aż po komina szczyt zatopione\\
    Wreszcie pożegnać przyszło piosenką\\
    Zadumane przy drodze kapliczki\\
    Mgłę tułaczkę ująć pod rękę\\
    Po raz ostatni górom się przyśnić
\end{text}
\begin{chord}
    G D\\
    C e\\
    G D\\
    e C D\\
    G D\\
    C e\\
    G D\\
    e C D

    G C G\\
    G C D\\
    D^{7}\\
    C D G
\end{chord}