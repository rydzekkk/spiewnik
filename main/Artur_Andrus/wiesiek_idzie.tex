\tytul{Wiesiek idzie}{sł. A. Andrus, muz. A. Grotowski}{Artur Andrus}
\begin{text}
Raz staruszka błądzącego w lesie.\\
Ujrzał lisek przywiędły i blady.\\
I pomyślał: 'Znowu idzie Wiesiek\\
Wiesiek idzie, nie ma na to rady'

\vin I podreptał do nory po ścieżce\\
\vin I oznajmił stanąwszy przed chatą.\\
\vin Swojej żonie lisicy Agnieszce\\
\vin Wiesiek idzie, nie ma rady na to.

Zaś lisica zmartwiła się szczerze\\
I machnęła łapkami obiema\\
Matko Boska! Bądź ostrożny, Jerzy\\
Wiesiek idzie, rady na to nie ma

\vin Może przybyć już dziś albo jutro\\
\vin Lub pojutrze, a może za tydzień\\
\vin Może nieźle przetrzepać nam futro\\
\vin Nie ma rady, Wiesiek, Wiesiek idzie

A był sierpień, pogoda prześliczna\\
I tętniło życie w zagajnikach.\\
Oprócz lisów nikt chyba nie myślał\\
O nadejściu Wieśka kłusownika

\vin Ale cóż, one żyły dość długo\\
\vin Łby na karkach miały nie od parady.\\
\vin I wiedziały, że prędzej czy później\\
\vin Wiesiek przyjdzie, nie ma na to rady
\end{text}
\begin{chord}
e A^7 e A^7\\
e A^7 h H^7\\
e A^7 e A^7\\
C H^7 e A^7

C D G e\\
C D G e\\
C D G e\\
C H^7 e A^7
\end{chord}
