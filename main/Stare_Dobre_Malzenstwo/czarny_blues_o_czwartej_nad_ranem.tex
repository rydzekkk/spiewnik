\tytul{Czarny blues o czwartej nad ranem}{sł. Adam Ziemianin, muz. Krzysztof Myszkowski}{Stare Dobre Małżeństwo}
\begin{text}
    \vin Czwarta nad ranem\\
    \vin może sen przyjdzie\\
    \vin może mnie odwiedzisz\\
    \vin Czwarta nad ranem\\
    \vin może sen przyjdzie\\
    \vin może mnie odwiedzisz

    Czemu cię nie ma na odległość ręki?\\
    Czemu mówimy do siebie listami?\\
    Gdy ci to śpiewam u mnie pełnia lata\\
    Gdy to usłyszysz będzie środek zimy

    Czemu się budzę o czwartej nad ranem\\
    I włosy twoje próbuję ugłaskać\\
    Lecz nigdzie nie ma twoich włosów\\
    Jest tylko blada nocna lampka,\\
    – łysa śpiewaczka

    Śpiewamy bluesa bo czwarta nad ranem\\
    Tak cicho żeby nie zbudzić sąsiadów\\
    Czajnik z gwizdkiem świruje na gazie\\
    Myślałby kto że rodem z Manhattanu

    \vin Czwarta nad ranem...

    Herbata czarna myśli rozjaśnia\\
    A list twój sam się czyta\\
    Że można go śpiewać za oknem mruczą bluesa\\
    Topole z Krupniczej

    I jeszcze strażak wszedł na solo\\
    Ten z Mariackiej Wieży\\
    Jego trąbka jak księżyc biegnie nad topolą\\
    Nigdzie się jej nie spieszy

    \vin Już piąta\\
    \vin Może sen przyjdzie\\
    \vin Może mnie odwiedzisz
\end{text}
\begin{chord}
    A\\
    cis\\
    D A\\
    E\\
    fis\\
    D E A

    A E\\
    fis cis\\
    D A\\
    D E

    A E\\
    fis cis\\
    D A\\
    D E\\
    fis
\end{chord}