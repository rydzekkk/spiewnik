%%
%% Author: bartek.rydz
%% 17.02.2019
%%
% Preamble
\tytul{W zakątku cmentarza}{sł. Bolesław Leśmian, muz. Krzysztof Myszkowski}{Stare Dobre Małżeństwo}
\begin{text}
    \vin Mają zmarli w niedzielę ten pośmiertny kłopot,\\
    \vin Że w obczyźnie cmentarza czują się - bezdomnie -\\
    \vin A lubią noc tę spędzać popod mgłą lub popod\\
    \vin Wiecznością, co się w jarach gęstwi nieprzytomnie.

    Maria z Bzówka - wygody wspomina izdebne,\\
    Słońce - w łóżku, wiatr - w sieni - i ogród macierzyn,\\
    Gdzie było tyle w radość uchodzących ścieżyn,\\
    A wszystkie takie - trafne i drzewom - potrzebne!...

    Żebrak, co się zadławił na śmierć krztyną chleba -\\
    Kijem niegdyś wędrownym obłędnie się babrze\\
    W nieodgadle błękitnym - pełnym Boga - chabrze,\\
    By zeń dla snu wiecznego wydłubać - źdźbło nieba.

    Mnich, co po to byt ziemski tłumił bez szemrania,\\
    By pędzić żywot wieczny w sposób nienaganny -\\
    Kreśli palcem na próchnie list do panny Anny\\
    Zyczac rychłego w kwiatach -zmartwychwstania.

    Panna Anna udaje, że jest - w bezżałobie\\
    I biorąc na kolana młodą mgłę - pieszczochę -\\
    Ukradkiem z pajęczyny tka zwiewną pończochę\\
    Dla brzozy, co tkwi boso na kochanka grobie.

    A opodal - mniej więcej naprzeciw rozstaju,\\
    We fraku bezrozumnie skąsanym przez szczura,\\
    Na czele kilku cieni żeńskiego rodzaju\\
    Nieboszczyk Madaleński - prowadzi mazura.
    
    \vin Mają zmarli w niedzielę...
\end{text}
\begin{chord}
    CFCFCFG\\
    CFCFG\\
    CFCFCFG\\
    e a G

    FGCFCF\\
    F G C\\
    FCFCFG\\
    F G C
    
    FGCFCF\\
    F G C\\
    FCFCFG\\
    F G C
    
    FGCFCF\\
    F G C\\
    FCFCFG\\
    F G C
    
    FGCFCF\\
    F G C\\
    FCFCFG\\
    F G C
    
    FGCFCF\\
    F G C\\
    FCFCFG\\
    F G C
\end{chord}