\tytul{Bieszczadzkie Anioły}{sł. A. Ziemianin, muz. K. Myszkowski}{Stare Dobre Małżeństwo}
\begin{text}
    Anioły są takie ciche, zwłaszcza te w Bieszczadach\\
    Gdy spotkasz takiego w górach, wiele z nim nie pogadasz\\
    Najwyżej na ucho ci powie, gdy będzie w dobrym humorze\\
    Że skrzydła nosi w plecaku, nawet przy dobrej pogodzie.

    Anioły są całe zielone zwłaszcza te w Bieszczadach\\
    Łatwo w trawie się kryją i w opuszczonych sadach\\
    W zielone grają ukradkiem, nawet karty mają zielone\\
    Zielone mają pojęcie, a nawet zielony kielonek

    \vin Anioły bieszczadzkie, bieszczadzkie anioły\\
    \vin Dużo w was radości i dobrej pogody\\
    \vin Bieszczadzkie anioły, anioły bieszczadzkie\\
    \vin Gdy skrzydłem cię trącą już jesteś ich bratem

    Anioły są całkiem samotne, zwłaszcza te w Bieszczadach\\
    W kapliczkach zimą drzemią, choć może im nie wypada.\\
    Czasem taki anioł samotny zapomni dokąd ma lecieć\\
    I wtedy całe Bieszczady mają szaloną uciechę

    \vin Anioły bieszczadzkie, bieszczadzkie anioły...

    Anioły są wiecznie ulotne, zwłaszcza te w Bieszczadach\\
    Nas też czasami nosi po ich anielskich śladach\\
    One nam przyzwalają i skrzydłem wskazują drogę\\
    I wtedy w nas się zapala wieczny bieszczadzki ogień.

    \vin Anioły bieszczadzkie, bieszczadzkie anioły...

\end{text}
\begin{chord}
    a G\\
    a e\\
    C G C F\\
    C G a e a

    a G\\
    a e\\
    C G C F\\
    C G a e a

    C G a\\
    C G a\\ 
    C G a\\
    C G a
\end{chord}