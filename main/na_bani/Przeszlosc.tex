%%
%% Author: bartek.rydz
%% 10.02.2019
%%
% Preamble
\tytul{Przeszłość}{sł. muz. J. Nohavica, tłum. T. Borkowski}{Na Bani}
\begin{text}
    \begin{tinyTwo}
    Jak gdy po nocy zapuka do drzwi nieproszony dziś gość\\
    Taki na ciebie za rogiem tuż czeka twój miniony los\\
    Buty skórzane, twoje ubranie, włosy ma też jak ty\\
    Z wolna kulejąc za tobą podąża aż krok zrówna z twym\\
    I powie zatem mnie masz\\
    Więc w sobie zważ czy wejść mi dasz\\
    Skoro mnie znasz\\
    Oto ja przeszłość twa

    Chustka w kieszeni, na której supełek zrobiłeś przed laty sam\\
    Już nie przypomni ci coś wtedy kupić koniecznie w sklepie miał\\
    To co po drodze straciłeś zniknęło jak pociąg w oddali znikł\\
    Tamta dziewczyna to niesie w plecaku i już nie odda ci\\
    Lecz mówi zatem mnie masz\\
    Więc w sobie zważ czy wejść mi dasz\\
    Skoro mnie znasz\\
    Oto ja przeszłość twa

    Drzewa są wyższe a trawa jest niższa na łące pleni się perz\\
    Tramwaj ma kolor jak chusta harcerza nosiłeś ją kiedyś też\\
    Tylko ta dumna ci głowa została by wciąż ją dźwigał twój kark\\
    Tamten konduktor co jeździł tu wtedy nie żyje już od lat\\
    Zatem mnie masz\\
    Więc w sobie zważ czy wejść mi dasz\\
    Skoro mnie znasz\\
    Oto ja przeszłość twa

    Gruszki na wierzbie nie rosną i z pokrzyw też w górę nie rośnie las\\
    Czego nie zjadłeś przedwczoraj i wczoraj to dzisiaj dojeść czas\\
    Solą migdały i grona dojrzałe które zanurzasz w nią\\
    Siedzisz przy stole a ona ci niesie gorący obiad już\\
    I mówi zatem mnie masz\\
    Więc w sobie zważ czy wejść mi dasz\\
    Skoro mnie znasz\\
    Oto ja przeszłość twa

    Połóż się w łóżku i obróć ku ścianie ona przytuli się\\
    Dziś cię odwiedzą ci, którzy istnieli, lecz już przeminęli gdzieś\\
    Ty po imieniu każdego zagadniesz bo wszystkich pamiętasz wciąż\\
    Rano się zbudzisz i w kłębek zwiniętą tuż obok znajdziesz ją

    Powie ci zatem masz mnie\\
    Zważ w sobie więc dałeś mi wejść\\
    Dość o mnie wiesz\\
    O mnie, przeszłości swej
    \end{tinyTwo}
\end{text}
\begin{chord}

\end{chord}