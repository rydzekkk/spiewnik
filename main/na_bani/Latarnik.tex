\section{Latarnik}
\begin{text}
    Niby morska, a w ląd wrośnięta latarnia\\
    Obrót światła w samotni lata odmierza\\
    Na zmianę to morze, to ziemię ogarnia\\
    Bez zmian w zawieszeniu na skraju wybrzeża

    I czyjeś tam życie wciąż koło zatacza\\
    Spiralą raz w górę, raz w dół dni mu biegną\\
    Latarnik -- strażnik na granicy światów\\
    Jak bóg ponad morzem, jednak mu podległy

    Duch nad wodami, bezwolny Światowid\\
    Mrok jasnowidzącą przebija smugą\\
    Złych kursów nie zmieni, choć może przepowie\\
    Nie panem losu jest, tylko sługą

    Nie on wprawia w ruch odległe okręty\\
    Okrążą go z dala, życie go omija\\
    A przy nim śmierć czeka w smętnych odmętach\\
    Ni morska to dola, ni ziemska -- niczyja

    Gdy buczków bezbrzeżnie smutny zew słychać\\
    Lotem kusi przepastna go wieży studnia\\
    We mgle ląd i woda, i świat cały znika\\
    Prosty życia horyzont dojrzeć najtrudniej

    Duch nad wodami, bezwolny Światowid\\
    Mrok jasnowidzącą przebija smugą\\
    Złych kursów nie zmieni, choć może przepowie\\
    Nie panem losu jest, tylko sługą
\end{text}