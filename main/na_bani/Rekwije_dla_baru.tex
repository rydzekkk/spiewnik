\section{Rekwije dla Baru na Stawach}
\begin{text}
Zachodzi słońce nad Placem Na Stawach\\
Już dawno bar podwoje zatrzasnął\\
Księżyc co na drugą zmianę wstawał\\
Usiadł na dachu przedmieścia i zasnął

Spóźniony gość sterczy na rogu\\
Wzrok wbił zgaszony w okno wystawy\\
Dusza w berecie w nim płacze bo widzi\\
Że nic nie zostało z Baru Na Stawach

Nic nie zostało tutaj z poezji\\
Może ten jeden poemat pijany\\
Ze śpiewem sunie nad pustą jezdnią\\
Zygzakiem płynąc od bramy do bramy

Nic nie zostało tutaj z poezji\\
Dawno ostatnie barowe czapy\\
Elektrowozem dziejów wywieźli\\
Zawisły na hakach prozy ochłapy

\hfill\break
W opustoszałym sklepie Na Stawach\\
Snują się duchy kufli po kątach\\
Umilkła drutem wiązana gitara\\
I szkło historii też ktoś wysprzątał

Nie będą więcej w kosmos ulatać\\
Słowa przy piwie odnajdywane\\
A na kamienic jesiennych dachach\\
Siądą gawrony już nie te same

Nic nie zostało tutaj z poezji\\
Cichcem wymknęły się samotne rymy\\
Rozpadł się wspomnień stary gołębnik\\
I skrzydła podciął czas cherubinom

Nic nie zostało tutaj z poezji\\
Bufet odleciał do raju bufetów\\
A w białym kitlu jak anioł rzeźnik\\
Pozostał tylko na pastwę poetów

Chociaż nie runął Na Stawach Bar\\
Osierocone pobladły ulice\\
I starsze nagle są twarze domów\\
Po ciemnych bramach chowa się życie

Zaszło już słońce nad Placem Na Stawach\\
I dawno Bar swe bramy zatrzasnął\\
Zamarła w pół kroku placowa zabawa\\
Pijany poemat pod ścianą zasnął
\end{text}
\begin{chord}
    cis\\
    A\\
    H\\
    fis gis Gis^7

    cis\\
    A\\
    H\\
    fis gis A\\
    gis A H\\
    E A\\
    fis H\\
    E A\\
    fis H

    E A\\
    D H\\
    E A\\
    fis H D\\
    fis G\\
    D gis Gis^7
\end{chord}