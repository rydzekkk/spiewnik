%%
%% Author: bartek.rydz
%% 16.02.2019
%%
% Preamble
\tytul{Droga za widnokres}{sł. T. Śliwiak, muz. M. Grechuta}{Marek Grechuta}
\begin{text}
    Nie na tym niebie, a i gwiazda nie ta,\\
    Weszła mi w szkodę, poraziła oczy.\\
    Wół stąpa ciężko, toczy się kareta.\\
    Patrzeć, a droga w gęsty las się toczy.

    W moim lesie, na moim południu,\\
    Wrzosy chodzą po rudych pagórkach.\\
    I już drzewa jesienią się trudnią\\
    Wiatr się stroi w ich opadłe pióra.

    U kapelusza powisają dzwonki,\\
    Aby okłamać skromność wesołością.\\
    Jakby żałobę pisać krojem czcionki,\\
    Co na wesele zwykła spraszać gości.

    Hej, hej, hej, hej, hej, hej,\\
    Hej, hej, hej, hej, hej, hej

    Gdzie jest ten kamień, za który nie sięga,\\
    Kolców i pokrzyw rozległy widnokres?\\
    Gdzie trochę ciepła, co u stóp przyklęka\\
    I rękawice osuszy mi mokre?

    W moim lesie wygasły już światła,\\
    Noc już w coraz większym płaszczu chodzi.\\
    Z boku kropli, ostatniej co spadła,\\
    Zimy biała gwiazda się urodzi.

    U kapelusza powisają dzwonki,\\
    Aby okłamać skromność wesołością.\\
    Jakby żałobę pisać krojem czcionki,\\
    Co na wesele zwykła spraszać gości.

    Hej, hej, hej, hej, hej, hej,\\
    Hej, hej, hej, hej, hej, hej

    W moim lesie białe ognie płoną,\\
    Rosną skrzydła śniegowej zawieji.\\
    W moim lesie świt cały ze szronu,\\
    W moim lesie jednak zima dmie.
\end{text}
\begin{chord}

\end{chord}