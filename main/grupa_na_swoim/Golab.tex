%%
%% Author: EL PROFESOR
%% %26.09.2020
%%
\tytul{Gołąb}{sł. muz. Arkadiusz Piechowski}{Grupa na Swoim}
\begin{text}
    W słuchawce zabrzmiał głos niewyraźnie:\\
    Spóźnię się trochę są korki straszne\\
    Nie tracąc czasu, poszedłem do parku,\\
    już się stęskniłem do śpiewu ptaków\\
    Z czarniutkiej teczki wyjąłem kanapkę,\\
    rozpiąłem kołnierzyk i siadłem na ławkę\\
    Widząc to gołąb stanął przede mną,\\
    dałbym kanapkę, ale mam jedną

    Gołąb nie poddał się ani trochę,\\ 
    patrzył w kanapkę wciąż wbitym wzrokiem\\
    Ja też patrzyłem bardzo zdziwiony,\\
    bo nie spotkałem takiej gołębiej persony\\
    Skrzydło zwichnięte, ogon zwichrzony,\\
    miał jedno oko, był przekrzywiony\\
    Przemówił do mnie lekko speszony:\\
    „Pozwoli pan prezes jestem gołąb pocztowy"

    Z dziada pradziada uprawiam ten fach\\
    latał mój ojciec i mój starszy brat\\
    Babcia ze łzami wspominała też\\ 
    o dziadku strąconym gdzieś pod Verdun\\
    Trochę przy pracy zwiedziło się,\\
    lecz nie pamiętam nazw tamtych miejsc\\
    Aż nadgodziny łapało się,\\
    latałem wszędzie gdzie dało się.

    Ponad dachami albo wśród chmur,\\
    nad jeziorami i szczytami gór\\
    W niejeden zachód w niejeden wschód,\\
    w ciepło południa, północy chłód\\
    Pod dziurawym dachem gdy padał deszcz\\ 
    spałem i śniłem o słońcu sen\\
    Z różnych chodników dziobałem chleb\\
    z niejednej kałuży napiłem się

    Przez niewyparzony gołębi dziób\\
    z kotem stoczyłem zaciekły bój\\
    Gdy od sierściuchów wyzwałem go\\
    oko straciłem stracił i on\\
    Był kiedyś ze mnie gołąb na schwał\\
    łatwo w miłosny wpadałem stan\\
    Co dziś zostało domyślam się\\
    jaja po świecie rozrzucone gdzieś

    Aby zapomnieć że zbity mam łeb\\
    lub serce głęboko zranione gdzieś\\
    Na dziecka znak wzbijałem się,\\
    za kawałek bułki pozując do zdjęć.\\
    Paw tam na dole rozkładał swój\\
    wachlarz z przepięknych olbrzymich piór\\
    Ja Bogu dziękuję zawsze i wszędzie\\
    że się wyklułem zwykłym gołębiem

    Dziś do latania nie mam już sił,\\
    lecz życie jest piękne prezesie, uwierz mi\\
    Nigdy nie przyszło by ci do głowy\\ 
    ile przeżyje zwykły gołąb pocztowy\\
    Słuchaj prezesie widzę że gołębie serce masz,\\ 
    leć gdzieś przed siebie, jeszcze jest czas\\
    Lecz nim na skrzydłach opuścisz tą ławkę,\\
    zostaw prezesie mi swą kanapkę


\end{text}
\begin{chord}
    C\\
    G\\ 
    a\\
    F\\ 
    C\\
    G\\
    a\\
    F

    C\\
    G\\ 
    a\\
    F\\ 
    C\\
    G\\
    a\\
    F

    C\\
    G\\ 
    a\\
    F\\ 
    C\\
    G\\
    a\\
    F

    C\\
    G\\ 
    a\\
    F\\ 
    C\\
    G\\
    a\\
    F

    C\\
    G\\ 
    a\\
    F\\ 
    C\\
    G\\
    a\\
    F

    C\\
    G\\ 
    a\\
    F\\ 
    C\\
    G\\
    a\\
    F

    C\\
    G\\ 
    a\\
    F\\ 
    C\\
    G\\
    a\\
    F
\end{chord}