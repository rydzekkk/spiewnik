%%
%% Author: bartosz.rydz
%% %28.05.2018
%%
\tytul{Deus ex machina}{sł.Agnieszka Osiecka, muz. Grzegorz Turnau}{Grzegorz Turnau}
\begin{text}
    Daj mi Panie rozpoznanie,\\
    żebym wiedział co jest co,\\
    czy mam wszystko mówić mamie,\\
    czy zachować to i to.

    To i to, ten i ta, deus ex machina.\\
    To i to, ten i ta, deus ex machina.

    Daj mi Panie rozpoznanie,\\
    kim ja jestem, kim ach kim,\\
    czy mam zostać leśnym drwalem,\\
    czy z wojskami zdobyć Rzym.

    To i to, ten i ta, deus ex machina.\\
    To i to, ten i ta, deus ex machina.

    Daj mi Panie rozpoznanie,\\
    czy ja z dobrych, czy ze złych,\\
    czy to twoje jest rozdanie,\\
    czy mam karty w rękach swych.

    To i to, ten i ta, deus ex machina.\\
    To i to, ten i ta, deus ex machina.

    Daj mi Panie rozpoznanie,\\
    czy mam oddać się na złom,\\
    czy dostawszy tęgie lanie,\\
    jeszcze nie pchać się pod prąd.

    To i to, ten i ta, deus ex machina.\\
    To i to, ten i ta, deus ex machina.

    Daj mi Panie rozpoznanie,\\
    czy szaleństwo jest tuż, tuż,\\
    czy to tyś miał Panie w planie,\\
    żeby nie żałować róż.

    To i to, ten i ta, deus ex machina.\\
    To i to, ten i ta, deus ex machina.

    Daj mi Panie rozpoznanie,\\
    czy zaryczy ranny łoś,\\
    kiedy przyjdzie już konanie,\\
    czy zapali światło ktoś.

    To i to, szyk i bzik, rapete, papete... pstryk.
\end{text}
\begin{chord}

\end{chord}