\tytul{Bieszczady (Tu w dolinach wstaje)}{sł. muz. Andrzej Starzec}{Andrzej Starzec}
\begin{text}
    Tu w dolinach wstaje mgłą wilgotny dzień\\
    Szczyty ogniem płoną, stoki kryje cień\\
    Mokre rosą trawy wypatrują dnia\\
    Ciepła, które pierwszy słońca promień da    

    \vin Cicho potok gada, gwarzy pośród skał\\
    \vin O tym deszczu, co z chmury trochę wody dał\\
    \vin Świerki zapatrzone w horyzontu kres\\
    \vin Głowy pragną wysoko, jak najwyżej wznieść

    Tęczą kwiatów barwny połoniny łan\\
    Słońcem wypełniony jagodowy dzban\\
    Pachnie świeżym sianem pokos pysznych traw\\
    Owiec dzwoneczkami cisza niebu gra

    \vin Cicho potok gada...

    Serenadą świerszczy, kaskadami gwiazd\\
    Noc w zadumie kroczy mroku ścieląc płaszcz\\
    Wielkim Wozem księżyc rusza na swój szlak\\
    Pozłocistym sierpem gasi lampy dnia

    \vin Cicho potok gada...
\end{text}
\begin{chord}
    e a\\
    D^7 G H^7\\
    e a\\
    D^7 G H^7

    G C D^7 G\\
    G C D^7 G\\
    G C D^7 G\\
    G C D^7 G

    e a\\
    D^7 G H^7\\
    e a\\
    D^7 G H^7\\
    \hfill\break

    e a\\
    D^7 G H^7\\
    e a\\
    D^7 G H^7
\end{chord}
