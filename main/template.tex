%%
%% Author: bartosz.rydz
%% %24.05.2018
%%
\tytul{tytuł}{autorzy}{zespół}
\begin{text}
    wers\\
    koniec strofy

    wers\\
    koniec strofy

    \hfill\break %dodatkowy pusty wers
	\textit{Capo III}\\ %tekst pochyły, oznaczenie kapo
    \vin refren\\
    \vin koniec refrenu
\end{text}
\begin{chord}

\end{chord}

Aby zmniejszyć czcionkę (tekstu i chwytów), w linijce po \begin trzeba wpisać:
\footnotesize{

natomiast w linijce przed \end:
}

Aby zwęzić tekst a poszerzyć chwyty, można użyć {textn} i {chordw} zamiast {text} i {chord}. Znaczenie to najpewniej TextNarrow i ChordWide.

Rysunki akordów: w sekcji chwytów wpisać \gtab{C}{X32010}. Pierwsze klamry to oznaczenie akordu, drugie to struny - opcjonalny próg i dwukropek, a dalej po kolei struny względem progu przed dwukropkiem (od basowej). Przykładowe akordy to \gtab{D}{XX0232}, \gtab{fis}{2:022000}.

Aby pominąć w chwytach refren (skrócony do jednej linijki) i dopisać chwyty do kolejnej zwrotki, należy w ostatniej linijce chwytów do zwrotki dać \\ na końcu, i w następnej linii dodatkowy pusty wers (\hfill\break). Przykład: panna_kminkowa.tex w folderze browar_zywiec.