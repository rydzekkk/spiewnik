\tytul{Kocham Cię zimo}{sł. i muz. Robert Marcinkowski}{Bez Zobowiązań}
\begin{textn}
    \begin{footTwo}
    \hfill\break
    \hfill\break
Kocham Cię Zimo zimowa prostoto myślenia.\\
Piękna i naga jak prawda jest rzeźba terenu.\\
Żadnych rozstajów dróg, krętych wątpliwości,\\
Skośną płaszczyzną dobrnę do Ciebie najprościej.

Cicho zapadam w międzypienne przestrzenie,\\
W śniegi dziewicze ucieczka, lot zapomnienia.\\
Z falą wdzięczności frunę na białe morze,\\
Na fale dolin ze zmierzchem się położę.

Proste dziś są me bezdroża i drogi krzywe\\
W szlak oczywisty prowadzą nad czasu igliwiem.\\
Z pozapętlanej logiki jeszcze się wywikłam,\\
Nim znów utonie w zdarzeniach ścieżka niezwykła.

Kocham Cię, Zimo, zimowa prostoto myślenia.\\
Piękna i naga jak prawda jest rzeźba terenu.\\
Żadnych rozstajów dróg, krętych wątpliwości,\\
Skośną płaszczyzną dobrnę do Ciebie najprościej.
\end{footTwo}
\end{textn}
\begin{chordw}
    \begin{footTwo}
    E^{2} C^{7+} H^{5+} | x3 E^{2} H^{7}

    E A^{2} C^{7+} H^{5+} E\\
    C^{7+} H^{5+} E A^{2} C^{7+} H^{5+} E\\
    C^{7+} H^{5+} A C^{7+} E^{2}\\
    A^{2} C^{7+} D^{2} E^{2}

    E A^{2} C^{7+} H^{5+} E\\
    C^{7+} H^{5+} E A^{2} C^{7+} H^{5+} E\\
    C^{7+} H^{5+} A C^{7+} E^{2}\\
    A^{2} C^{7+} D^{2} E^{2}

    E A^{2} C^{7+} H^{5+} E\\
    C^{7+} H^{5+} E A^{2} C^{7+} H^{5+} E\\
    C^{7+} H^{5+} A C^{7+} E^{2}\\
    A^{2} C^{7+} D^{2} E^{2}

    E A^{2} C^{7+} H^{5+} E\\
    C^{7+} H^{5+} E A^{2} C^{7+} H^{5+} E\\
    C^{7+} H^{5+} A C^{7+} E^{2}\\
    A^{2} C^{7+} D^{2} E^{2}
\end{footTwo}
\end{chordw}