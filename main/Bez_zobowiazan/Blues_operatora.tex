\tytul{Blues Operatora}{s?. muz. K.R.Marcinkowscy}{Robert Marcinkowski}
\begin{text}
    stalowa wieża, kij wsadzony w miasto\\
    tu jestem kimś, a jak będę chciał, to skoczę

    Siedzę sobie w dźwigu, dźwig na luźnym biegu jest.\\
    Siedzę sobie w dźwigu, dźwig na luzie i ja też.\\
    W dole mnie szukają, a tam wcale nie ma mnie.

    Dziecko się topiło - teraz chcą mi medal dać.\\
    O hak zaczepiło - teraz chcą mi, psia ich mać.\\
    Wzruszam ja ramieniem, co sześćdziesiąt metrów ma.

    Vis a vis panienka: buzia w ciup, na głowie kok.\\
    Na dziesiątym piętrze sterczy w oknie całą noc.\\
    Może bym podjechać, ale szyny idą w bok.

    Prezes jest na dole. Chce mi podać bratnią dłoń.\\
    On ma tam swój stołek. Ja na wieży mam swój tron.\\
    Na stalowym haku zwisa mi szesnaście ton.

    stalowa wieża, kij wsadzony w miasto\\
    tu jestem kimś, a jak będę chciał, to skoczę\\
    skoczę na piwo, a wyzwolony blues\\
    poleci w dół jak ptak, odlany z betonu
\end{text}
\begin{chord}
    E^7\\
    E^7

    E G A G x2\\
    EGAG(H)x2\\
    A H G A\\
    E G A H\\
    E G A G x2\\
    EGAG(H)x2\\
    A H G A\\
    E G A H\\
    EGA G x2\\
    EGAG(H)x2\\
    A H G A\\
    E G A H\\
    EGA G x2\\
    EGAG(H)x2\\
    A H G A\\
    E G A H
\end{chord}