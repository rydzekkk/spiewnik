%%
%% Author: bartek.rydz
%% 06.02.2019
%%
% Preamble
\tytul{Wypa Truję}{sł. muz. Robert Marcinkowski}{Robert Marcinkowski}
\begin{text}
    \small{
    Wypa-truję kiedy ta – dziwnie na mnie wpływa mgła\\
    i próbuję truć we mgle: we mgle Wypa * truje się.

    Droga idzie. Idzie, ginie, ginie i nie widzę, czy nie\\
    zginie po tym czynie, czy się objawi * droga mi?

    Gdy pod śniegiem czujesz pień – nie wydzwaniaj nerą weń.\\
    Gdy cię w plery smaga pień – dupoślizgu * trasę zmień.

    Wypa truj! Czy Wypa trujesz? Gdy rozterka cię nurtuje –\\
    trując Wypa – okiem łyp: może już zatruty * zatruty Wyp?

    Wypa truję, truję Wypa, nie wypada truć, nie wypa...\\
    ...trując Wypa okiem łyp: Może już zatruty * zatruty Wyp?\\
    bis: ... / Może już zatruty *

    Uwaga: w miejscach oznaczonych* należy wydać dźwięk
    zbliżony do indiańskiego słowa „uwgh”.
    }
\end{text}
\begin{chord}
    \small{
    E^{73} B A E^{73}\\
    E^{73} B A E^{73}

    E^{73} B A E^{73}\\
    E^{73} B A E^{73}

    E^{73} B A E^{73}\\
    E^{73} B A E^{73}

    E^{73} B A E^{73}\\
    E^{73} B A E^{73}

    E^{73} B A E^{73}\\
    E^{73} B A E^{73}
    }
\end{chord}