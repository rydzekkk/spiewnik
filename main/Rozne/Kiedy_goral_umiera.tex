%%
%% Author: bartek.rydz
%% 12.02.2019
%%
% Preamble
\tytul{Kiedy góral umiera}{sł. muz. P. Kasperczyk}{Babsztyl}
\begin{text}
    \begin{smallTwo}
    Kiedy góral umiera, to góry z żalu sine\\
    Pochylają nad nim głowy jak nad swoim synem\\
    Las w oddali szumi mu odwieczną pieśń bukową\\
    A on długo sposobi się przed najdalszą drogą

    Kiedy góral umiera, to nikt nie układa baśni\\
    Tylko w niebie roziskrzonym mała gwiazdka gaśnie\\
    Głowę jeszcze raz uniesie, do góry do nieba\\
    By pożegnać góry swe, by im coś zaśpiewać

    \vin Góry moje, wierchy moje, otwórzcie swe ramiona\\
    \vin Niech na miękkim z mchu posłaniu cichuteńko skonam\\
    \vin Ojcze mój, halny wietrze, powiej ku północy\\
    \vin Ciepłą drżącą swoją ręką zamknij zgasłe oczy\\
    \vin Bym mógł w ziemię wrosnąć\\
    \vin Strzelić potem do słońca smreczyną\\
    \vin I na zawsze szumieć już\\
    \vin Nad swoją dziedziną

    Kiedy góral umiera, to nikt nad nim nie płacze\\
    Siedzi, czeka aż kostucha w okno zakołacze\\
    Ziemia twardą szorstką ręką tuli go do siebie\\
    By na zawsze zostać mógł pod góralskim niebem
\end{smallTwo}
\end{text}
\begin{chord}
    \begin{smallTwo}
    D D^{7}\\
    G D\\
    e G D\\
    e G D

    D D^{7}\\
    G D\\
    e G D\\
    e G D

    D e\\
    G D\\
    D e\\
    G D\\
    e\\
    G D\\
    e\\
    G D
\end{smallTwo}
\end{chord}