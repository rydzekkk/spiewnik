\tytul{Beskidzki sklep}{sł. muz. Andrzej Wierzbicki}{Andrzej Wierzbicki}
\begin{text}
    A w Beskidzie gra muzyka, tańczą święci krasnozłoci,\\
    z baniek cerkwi piją wino poczerniałe wiedźmy złockie.\\
    Na przypieckach mruczą koty, w kącie żarna jęczą rzewnie,\\
    tylko jeden dach spokojnie, cicho sterczy w sennym niebie.

    \vin Pójdziesz prosto polną drogą,\\
    \vin tam gdzie cerkwi piersi złote,\\
    \vin przez spieniony przejdziesz potok,\\
    \vin po napiętych strunach mostu,\\
    \vin przy kapliczce wejdziesz ścieżką\\
    \vin na kamienne stopnie trzy\\
    \vin drzwi otworzą się z szelestem – i...

    \vin Tam na półkach dziecięce marzenia karmelkami płoną,\\
    \vin tam się chmiel z dębiną żeni, będzie złoty owoc,\\
    \vin w workach z kaszą kichają myszy, naftą pachnie chleb,\\
    \vin a sklepikarz basem dyszy, zaciągając z ruska śpiewnie,\\    
    \vin a sklepikarz basem dyszy, zaciągając z ruska śpiewnie.\\
    \vin To mój sklep – beskidzki sklep.

    A w Beskidzie weselisko, roztańczone korowody,\\
    lato wkłada swoje ręce za pazuchę wiośnie młodej.\\
    Wieczór w okna sypie przędzę, na firankach białych siada,\\
    tylko jedna szyba pusta spoza krat z gwiazdami gada.

    \vin Pójdziesz prosto...

    \vin Tam na półkach...
\end{text}
\begin{chord}
    D G D C\\
    F C G A\\
    D G D C\\
    F C G A

    e\\
    D\\
    e\\
    D\\
    e\\
    D\\
    C A

    D G D G\\
    D G D G\\
    C D A\\
    C A\\    
    C A\\
    C G A D
\end{chord}